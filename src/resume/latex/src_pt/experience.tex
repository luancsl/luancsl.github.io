%-----------EXPERIENCE-----------%
\section{Experiência}
\resumeSubHeadingListStart

\resumeSubheading
{Laboratório de Computação Embarcada e Tecnologias Industriais (LACETI-CIN)}{Nov 2022 -- Presente}
{Engenheiro de Software Sênior}{Recife, PE}
\resumeItemListStart
\resumeItem{Implementou data lake IoT (76k dispositivos) com MQTT, Kafka, Spark; integrou ML/analytics; gerenciou com Terraform; processamento em tempo real/batch.}
\resumeItem{Projetou arquitetura de microserviços em Kubernetes com Istio como service mesh, adicionando observabilidade via Prometheus, Grafana para dashboards e Jaeger para rastreamento distribuído.}
\resumeItem{Estabeleceu pipeline de CI/CD usando CircleCI e ArgoCD, aplicando práticas MLOps com MLflow para versionamento de experimentos e KServe para servir modelos em Kubernetes.}
\resumeItem{Liderou uma equipe multifuncional de 6 pessoas na implementação de uma segunda versão de um sistema de execução de linha de manufatura (MES) utilizando React e Node.js, resultando em um aumento de 25\% no desempenho.}
\resumeItemListEnd

\resumeSubheading
{Laboratório de Computação Embarcada e Tecnologias Industriais (LACETI-CIN)}{Jun 2021 -- Nov 2022}
{Engenheiro de Software}{Recife, PE}
\resumeItemListStart
\resumeItem{Desenvolveu sistema MES para linha de produção de baterias usando React, GraphQL e Node.js. Criou interfaces responsivas, implementou APIs e serviços GraphQL, e integrou dados de produção em tempo real.}
\resumeItem{Estabeleceu testes automatizados para aplicações front-end, alcançando 90\% de cobertura. Usando Cypress para testes end-to-end e Jest para testes unitários, estabeleceu uma suíte que garantia a integridade funcional da aplicação.}
\resumeItem{Otimização da sincronização de dados entre sensores da linha de produção e nuvem usando uma solução de cache com Redis e Node.js. A solução reduziu a latência na transmissão de dados e melhorou a confiabilidade da sincronização.}
\resumeItemListEnd

\resumeSubheading
{Laboratório Multidisciplinar de Tecnologias Sociais (LMTS)}{Mai 2019 -- Jun 2021}
{Engenheiro de Software Júnior}{Garanhuns, PE}
\resumeItemListStart
\resumeItem{Projetou 'VacinaGaranhuns', um sistema React/Google Maps para gerenciar a vacinação contra COVID-19. Incluiu agendamento online, geolocalização e vacinação domiciliar, otimizando a campanha para 12.000 pessoas.}
\resumeItem{Liderou o desenvolvimento de um sistema de irrigação inteligente com React Native e Node.js, integrando APIs meteorológicas e dispositivos IoT.}
\resumeItem{Utiliza análises avançadas para cálculo preciso de evapotranspiração, resultando em 62\% de redução no consumo de água para pequenos agricultores.}
\resumeItemListEnd

\resumeSubHeadingListEnd